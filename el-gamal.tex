\documentclass[12pt, a4paper]{article}
\setlength{\topmargin}{-2cm}
\setlength{\oddsidemargin}{0cm}
\setlength{\textheight}{24cm}
\setlength{\textwidth}{16cm}
\usepackage{graphicx}
\usepackage{color}
\usepackage{listings}
\usepackage{sectsty}
\usepackage{helvet}
\usepackage{color}
\usepackage{xcolor}
\usepackage{amsmath}
\usepackage{amssymb}
\usepackage{hyperref}
\usepackage{polynom}
\usepackage{fancyhdr}
\usepackage{amsfonts}
\usepackage{soul}
\usepackage{amssymb}
\usepackage{ulem}
\usepackage{listings}
\usepackage{caption}
\DeclareCaptionFont{white}{\color{white}}
\DeclareCaptionFormat{listing}{\colorbox{gray}{\parbox{\textwidth}{#1#2#3}}}
\captionsetup[lstlisting]{format=listing,labelfont=white,textfont=white}
\allsectionsfont{\sffamily}

\begin{document}
\pagestyle{fancy}
\fancyhead[L]{El-Gamal Encryption}
\fancyhead[R]{David Leonard}

\vspace{0.5cm}\hspace{3.0cm}{\Large \textbf{Breaking El-Gamal Encryptions}}
\vspace{0.25cm}

\hspace{6.0cm}{\normalsize September 5, 2013}

\vspace{0.25cm}
\hspace{6.2cm}{\normalsize David Leonard}

\vspace{0.25cm}
\hspace{5.1cm}{\normalsize \texttt{DrkSephy1025@gmail.com}}

\vspace{0.7cm}



$$(\mathbb{Z}_{p-1}) = \mathbb{Z}_{q} X \mathbb{Z}_{2}$$

$$((r_0, r_1), (m_0), (m_1) + a(r_0, r_1)) = (g^r, (g^a)^rm)$$
             \hspace{5.0cm}$\uparrow$$\mathbb{Z}_{q}$ X $\mathbb{Z}_{2}$ \hspace{3.5cm}   $\uparrow$$(\mathbb{Z}_{p})^*$\newline
             
             
$$ M \in {m_0, m_1}$$
  \hspace{6.5cm} $\uparrow$$(\mathbb{Z}_{p})^*$	$\uparrow$(mod $q$,mod $2$)

$$q(m_0, m_1) = ((\sout{qm_0}, qm_1)$$


$$ 1 \longrightarrow 0 \longrightarrow (0,0) $$


$$ (\mathbb{Z}_{p})^* \longrightarrow (\mathbb{Z}_{p-1}) \longrightarrow \mathbb{Z}_{q} X \mathbb{Z}_{2} $$

In $\mathbb{Z_{q}} X \mathbb{Z}_{2}$, we isolate $m_1$:

$$q(m_0, m_1) = (0, m_1)$$

We know that the equivalent form of the above expression in the multiplicative group,  \newline

$(\mathbb{Z}_{p})^*$ is exponentiation.

$$\therefore  M^q | \mathbb{Z}_{q} X \mathbb{Z}_{2}$$

We now see that $m_1 = 1$ $\Leftrightarrow$ $m^q$ mod $p$ $\ne 1$

$$((r_0, r_1), (m_0, m_1) + a(r_0, r_1)) = (g_r, (g^a)^rm)$$

Using the above information, we arrive at the following set of linear equations of M:

$$(r_1, m_1 + ar_1)$$

$$y = x + ar_1 \longrightarrow M = (M_0, M_1)$$
$$y - ar_1 = x \longrightarrow M' = (M_{0'}, M_{1'}$$

We know how to compute $r_1$ and $a$ from: 

$$g^a = a = (a_0, a_1)$$

The adversary $A$ can now efficiently compute M! 

\end{document}
